%*******************************************************************************
\part{Discussions and Conclusions}
\chapter{Current Status}

\chapter{Critical Evaluation}
%TODO Search for FUTURE tags
%TODO pandas looks like Frontier on steroids...
%TODO Mention FeatureForge?
Decision trees also cannot promise optimality...
...try random forests in future
...would also like to experiment with other algorithms (even black box) and frameworks...

%TODO Cite KSelect
...when choosing best parameters...
Alternatively could have used \textbf{SelectKBest}...
best practice...\citep{sl:tips}

...Section~\ref{sec:additional-libs}
The \textbf{SciPy Stack} also includes the "Python Data Analysis Library"
(\textbf{pandas}), which is designed to provide tools to supplement the Python
environment, allowing users to perform analysis in Python instead of having to
switch to a more analysis focused language such as R.

...at first glance appears to be a far more developed Frontier...
...however it should be considered that Frontier was designed to complement the
machine learning experiments and provides 

...whilst also providing the \textbf{AbstractReader} framework to allow users to
quickly define methods to read in their own data regardless of how esoteric or
cryptic the file format is...


Whilst a 2003 paper\citep{kendal2003exponential}\footnote{Titled \textit{An Exponential
Dispersion Model for the Distribution of Human Single Nucleotide
Polymorphisms}\citep{kendal2003exponential}} that analysed the spread of SNPs
across the human genome to assert whether variants could be modelled with a
statistical distribution, divided chromosomes in to equally sized bins (a
concept akin to candidates of uniform length)

\chapter{Conclusions}

