\chapter{Implementation}
\ifpdf
    \graphicspath{{Chapter3/Figs/Raster/}{Chapter3/Figs/PDF/}{Chapter3/Figs/}}
\else
    \graphicspath{{Chapter3/Figs/Vector/}{Chapter3/Figs/}}
\fi

%%%%%%%%%%%%%%%%%%%%%%%%%%%%%%%%%%%%%%%%%%%%%%%%%%%%%%%%%%%%%%%%%%%%%%%%%%%%%%%
\section{Introduction}

%TODO Better description of Frontier
This chapter introduces \textbf{Frontier}; the main programming effort for this
part of the project. Frontier is a Python package that serves as a data manager,
providing both interfaces to read inputs into structures in memory and to
retrieve them in formats acceptable to a machine learning framework.

\subsection{Concepts}

\subsubsection{Classes and Labels}

\subsubsection{API Access}

\subsubsection{Cross Validation}

%%%%%%%%%%%%%%%%%%%%%%%%%%%%%%%%%%%%%%%%%%%%%%%%%%%%%%%%%%%%%%%%%%%%%%%%%%%%%%%
\section{Implementation}


...provides a class to read from the "bamcheckr'd files" as seen in
Appendix~\ref{app:bamcheckr}... as well as the auto\_qc output matrix briefly
demonstrated in Appendix~\ref{app:aqc_matrix}...

...a major refactoring to remove hard-coded classes from \textbf{Frontier} which
enable it to be used as a more general purpose tool; if we were to add another
class label, the definition would merely need to be included to the CLASSES
(Listing~\ref{fig:FrontierClasses}) variable passed when the Statplexer is
constructed. But use is therefore not merely limited to our problem but rather
any machine learning problem where you'd like to simplify your interactions
which a very large dataset.

\begin{listing}[H]
    \caption[FrontierClasses]{Class definitions for auto\_qc as passed to Frontier}
    \label{fig:FrontierClasses}
    \begin{minted}[mathescape,
                %linenos,
                numbersep=5pt,
                gobble=8,
                frame=lines,
                framesep=2mm]{python}
        CLASSES = {
                "pass": {
                    "class": ["pass"],
                    "names": ["pass", "passed"],
                    "code": 1,
                },
                "fail": {
                    "class": ["fail"],
                    "names": ["fail", "failed"],
                    "code": -1,
                },
                "warn": {
                    "class": ["warn"],
                    "names": ["warn", "warning"],
                    "code": 0,
                },
        }
    \end{minted}
\end{listing}


\subsubsection{Cross Validation}
...K fold cross validation
...stratified cross validation...

\subsubsection{Confusion Matrices}
"Normal" confusion matrix and "Warnings" confusion matrix...

\subsection{Contributions to bamcheckr}
\begin{minted}[mathescape,
               %linenos,
               numbersep=5pt,
               gobble=4,
               frame=lines,
               framesep=2mm]{r}
    install.packages("devtools")
    library(devtools)

    # Install directly from github repository
    install_github("samstudio8/seq_autoqc", subdir="bamcheckr")

    # Install from local directory
    install("/home/sam/Projects/seq_autoqc/bamcheckr")

\end{minted}
Takes 5.5s on average, 16.1s with ratio due to inefficient implementation
for overlapping\_base\_duplicate\_percentage
re-wrote in Python...


...R CMD BATCH issue

...Fixed a graph plotting failure.
...Writing additional routines...

\subsection{SelectKBest}
* incorrect degrees of freedom
* warnings: /usr/lib64/python2.7/site-packages/sklearn/feature\_selection/univariate\_selection.py:256: RuntimeWarning: invalid value encountered in divide, causing NaN
* Replaced univariate\_selection with version from master
* needed use force np.float64
...actually data was 0... gg :(

%%%%%%%%%%%%%%%%%%%%%%%%%%%%%%%%%%%%%%%%%%%%%%%%%%%%%%%%%%%%%%%%%%%%%%%%%%%%%%%
\section{Testing}
