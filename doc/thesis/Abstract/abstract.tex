% ************************** Thesis Abstract *****************************
\begin{abstract}

Over the past few years advances in genetic sequencing hardware have introduced
the concept of massively parallel DNA sequencing; allowing potentially billions
of chemical reactions to occur simultaneously, reducing both time and cost
required to perform genetic analysis\citep{HMG}. However, these "next-generation"
processes are complex and open to error\citep{Illumina}, thus quality control is
an essential step to assure confidence in any downstream analyses performed.

During sample sequencing a large number of quality control metrics are generated
to determine the quality of the reads from the sequencing hardware itself. At the
Wellcome Trust Sanger Institute, the automated QC system currently relies on hard
thresholds to make such quality control decisions with individual hard-coded
values on particular metrics determining whether a lane has reached a level that
requires a warning, or has exceeded the threshold and failed entirely. Whilst this
does catch most of the very poor quality lanes, a large number of lanes are
flagged for manual inspection at the warning level; a time consuming task which
invites inefficiency and error.

In practise most of these manual decisions are based on inspecting a range of
diagnostic plots which suggests that a machine learning classifier could
potentially be trained on the combinations of quality control statistics
available to make these conclusions without the need for much human intervention.

\end{abstract}
