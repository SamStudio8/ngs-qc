%*****************************************************************************************
\part{Identification of Qualitative Sample Properties}
\chapter{Introduction and Motivation}

%%%%%%%%%%%%%%%%%%%%%%%%%%%%%%%%%%%%%%%%%%%%%%%%%%%%%%%%%%%%%%%%%%%%%%%%%%%%%%%
\section{Introduction}

This part of the project can be outlined as follows:

\begin{itemize}
    \item Selection of a genomic region...
\end{itemize}

\subsection{Why Goldilocks?}
%TODO Tidy dumped blog pasta
%TODO What miscomm?

Having recovered from the miscommunication that led me to attempt to find
reverse strands in VCF files with already known errors, I’ve been making
performance improvements to the script that finds candidate genomic regions.

The task poses an interesting problem in terms of complexity and memory, as the
human genome is over three billion bases long in total which can easily lead to
data handling impracticalities.

For the next step of the project we are looking to document what effects quality
has on analysis that occurs downstream from sequencing, for example: variant
calling - the process of identifying bases in a sample that differ from the
reference genome (this is a little simple as you also need to discern as
whether the difference is actually a polymorphism or not).

To investigate this I’ll be performing leave-one-out analysis on the
whole-genome data we have, consisting of leaving a lanelet out, performing
variant calling and then comparing the result of the left-one-out variant call
and the corresponding variant call on our “SNP chip” data.

The basic idea is to answer the question of what constitutes “good” or
“bad” lanelet quality? Does leaving out a lanelet from the full-sequence data
lead to variant calls that better match the SNP chip data, or cause the
correspondence between the two sets to decrease? In which case, having
identified such lanelets, can we look back to the quality parameters we’ve been
analysing so far and find whether they have something in common in those cases?

If we can, these parameters can be used to identify “good” or “bad” lanelets
straight out of the machine. We know that lanelets that exhibit these quality
variables will go on to improve or detriment analysis.

However, variant calling is both computationally and time intensive.  Whilst
the Sanger Institute have significant computing resources available, my
dissertation has an end date and with the time I have we must focus the
leave-one-out analysis on a subsection of the whole genome.

It is for this reason we’re looking for what Josh at Sanger described as a
“representative autosomal region”. The candidate must not have too many
variants, or too few: a “Goldilocks genome”.

%%%%%%%%%%%%%%%%%%%%%%%%%%%%%%%%%%%%%%%%%%%%%%%%%%%%%%%%%%%%%%%%%%%%%%%%%%%%%%%
\chapter{Materials and Methods}
\section{Input Data and Format}
\subsection{Variant Call Format}

\begin{minted}[mathescape,
            %linenos,
            numbersep=5pt,
            gobble=4,
            frame=lines,
            framesep=2mm]{bash}
    # Install
    git clone htslib
    git clone bcftools
    make #(requires htslib to be above samtools/bcftools dir)
    sudo cp bcftools usr/local/bin

    # Tabix
    # Download from sourceforge
    # http://sourceforge.net/projects/samtools/files/tabix/
    make
    sudo cp tabix /usr/local/bin

    # Generate an index (vcfidx) (not necessary)
    vcftools --gzvcf cd.ichip.vcf.gz

    # Query VCF (can be done with awk, cut etc.)
    vcf-query cd.ichip.vcf.gz -f '%CHROM:%POS\t%REF\t%ALT\n'

    # ! A faster alternative to using vcftools exists in bcftools ! #
    bcftools query -f '%CHROM:%POS\t%REF\t%ALT\n' cd-seq.vcf.gz > cd-seq.vcf.gz.q
    # Complains about lack of tabix index but still makes the file...

    # Index with tabix
    # "The input data file must be position sorted and compressed by bgzip 
    # which has a gzip(1) like interface"
    tabix -p vcf file.vcf.gz
    diff cd-seq.vcf.gz.tbi diff cd-seq.vcf.gz.tbi.sanger
    # Shows no difference :)
    # http://samtools.sourceforge.net/tabix.shtml

    # http://vcftools.sourceforge.net/perl_module.html
    # http://www.biostars.org/p/51076/
    # http://vcftools.sourceforge.net/htslib.html#query
\end{minted}


%%%%%%%%%%%%%%%%%%%%%%%%%%%%%%%%%%%%%%%%%%%%%%%%%%%%%%%%%%%%%%%%%%%%%%%%%%%%%%%
\chapter{Implementation}
\section{Goldilocks}
\subsection{Introduction}

As I discussed previously, on the hunt for my Goldilocks genomic region, it is
important to consider both time and memory as it is simple to deliver a poor
performing solution.

Searching for candidates regions over the entire genome at once is probably
unwise. Luckily, since our candidate must not span chromosomes (the ends of
chromosomes are not very good for reads) then we can easily yield a great
improvement from processing chromosomes individually.

The process is to extract the locations of all the variants across the genome
from the SNP chip VCF files (these files list the alleles for each detected
variant for each sample), load them in to some sort of structure, “stride” over
each chromosome (with some stride offset) and finally list the variants present
between the stride start and stride start plus the desired length of the
candidate. These are our regions.

Due to the use of striding, one does not simply walk the chromosome. A quick
and dirty solution would be to just look up variants in a list:

\begin{minted}[mathescape,
            %linenos,
            numbersep=5pt,
            gobble=4,
            frame=lines,
            framesep=2mm]{python}
    for chromosome in autosomes:
        for start in range(1, len(chromosome), STRIDE):
            for position in range(start, start+LENGTH):
                if position in variant_position_list:
                    # Do something...
\end{minted}

This is of course rather dumb. Looking up a list has $\mathcal{O}$(n) performance
where n is the number of variants (and there are a lot of variants). Combining
this with the sheer number of lookups performed, with the default parameters the
list would be queried half a billion times for the first chromosome alone.

You could improve this somewhat by dividing the variant\_position\_list in to a
list of variants for each chromosome, so at least n is smaller. It is pretty
doubtful that this would make any useful impact given the number of lookups.
I’m happy to say I bypassed this option but thought it would be fun to consider
its performance.

A far more sensible solution is to replace the list with a dictionary
whose lookup is amortized to a constant time. The number of lookups is still
incredibly substantial but a constant lookup is starting to make this sound
viable. Using the variant positions as keys of a dictionary (in fact let us use
the previous minor suggestion and have a dictionary for each chromosome) we
have:

\begin{minted}[mathescape,
            %linenos,
            numbersep=5pt,
            gobble=4,
            frame=lines,
            framesep=2mm]{python}
    for chromosome in autosomes:
        for start in range(1, len(chromosome), STRIDE):
            for position in range(start, start+LENGTH):
                if position in variant_position_dict[chromosome]:
                    # Do something...
\end{minted}

This still takes half an hour to run on my laptop though, surely there
is a better way.  I wondered if whether dropping the lookup would improve
things. Instead, how about an area of memory is allocated to house the current
chromosome and act as a variant “mask” -- essentially an array where each element
is a base of the current chromosome and the value of 1 represents a variant at
that position and a 0 represents no variant.

\begin{minted}[mathescape,
            %linenos,
            numbersep=5pt,
            gobble=4,
            frame=lines,
            framesep=2mm]{python}
    for chromosome in autosomes:
        chro = np.zeros(len(chromosome), np.int8)
        for variant_loc in snps_by_chromosome[chromosome]:
            chro[variant_loc] = 1

        for start in range(1, len(chromosome), STRIDE):
            for position in range(start, start+LENGTH):
                if chro[position] == 1:
                    # Do something...
\end{minted}

We of course have to initially load the variants in to the chromosome array,
but this need only be done once (per chromosome) and is nothing compared to the
billions of lookups of the previous implementation.

Rather to my surprise this was slow  (maybe if I have time I should
check how it compared to looking up with a list). Was it the allocation of
memory? Perhaps I had run out of RAM and most of the work was going in to
fetching and storing data in swap?

I switched out the comparison (== 1) step for the previous dictionary based
lookup and the performance improved considerably. What was going on? There must
be more to looking at a given element in the numpy chromosome array, but what?

After a brief spell of playing with Python profilers and crashing my laptop by
allocating considerably more memory than I had with numpy.zeroes, I read the
manual (!) and discovered that numpy.zeroes returns an ndarray which uses
“advanced indexing” even for single element access, effectively copying the
element to a Python scalar for comparison.

That’s a lot of memory action.

It then occurred to me that we’re interested in just how many variants are
inside each region and our chromosome is handily encapsulated in a numpy array.
Why don’t we just sum together the elements in each region? Remember variant
positive base positions are 1 and 0 otherwise, useful. So the work boils down
to some clever vector mathematics calculated by numpy.

We don’t even lose any detail because the actual list of variants inside a
region can be recovered in a small amount of time just given the start and end
position of the region.

\begin{minted}[mathescape,
            %linenos,
            numbersep=5pt,
            gobble=4,
            frame=lines,
            framesep=2mm]{python}
    for chromosome in autosomes:
        chro = np.zeros(len(chromosome), np.int8)
        for variant_loc in snps_by_chromosome[chromosome]:
            chro[variant_loc] = 1

        for start in range(1, len(chromosome), STRIDE):
            num_variants = np.sum(chro[start:start+LENGTH])
            # Do something...
\end{minted}

With conservative testing this runs at least 60 times faster than before, the
entirety of the human genome can be analysed for candidate regions in less than
twenty seconds (with the first few seconds taken reading in all the variant
locations in the first place).

Ignoring empty regions etc.

\begin{minted}{bash}
    bsub -o ../../../goldilocks/joblog/samtools_mpileup.%J.o -e ../../../goldilocks/joblog/samtools_mpileup.%J.e -G hgi -J "samtools_mpileup" -M1000 -R "select[mem>1000] rusage[mem=1000]" bash -c 'samtools mpileup -b ../goldilocks-3:46000001-47000000.fofn -g -I -f /lustre/scratch113/resources/ref/Homo_sapiens/1000Genomes_hs37d5/hs37d5.fa > ../../../goldilocks/all.withref.bcf'
\end{minted}
% Needed -q long to submit to 48 hour queue instead of 12

Memory Leaks...
\begin{minted}{txt} 
C symbol: bam_destroy1

  File          Function          Line
  0 bam.c         bam_fetch           64 bam_destroy1(b);
  1 bam_mate.c    bam_mating_core    269 bam_destroy1(b[0]);
  2 bam_mate.c    bam_mating_core    270 bam_destroy1(b[1]);
  3 bam_md.c      bam_fillmd         383 bam_destroy1(b);
  4 bam_rmdup.c   dump_best           41 bam_destroy1(stack->a[i]);
  5 bam_rmdup.c   bam_rmdup_core     174 bam_destroy1(b);
  6 bam_rmdupse.c __free_elem         14 #define __free_elem(p) bam_destroy1((p)->data.b)
  7 bam_rmdupse.c bam_rmdupse_core   157 bam_destroy1(b);
  8 bam_stat.c    bam_flagstat_core   44 bam_destroy1(b);
  9 bam_tview.c   sam_fetch           47 bam_destroy1(b);
  a bamshuf.c     bamshuf             80 bam_destroy1(b);
  b bamshuf.c     bamshuf            111 bam_destroy1(a[j].b);
  c padding.c     bam_pad2unpad      318 bam_destroy1(b);
  d phase.c       dump_aln           341 bam_destroy1(b);
  e sam.c         sampileup           61 bam_destroy1(b);
  f sam_view.c    main_samview       356 bam_destroy1(b);
  g sam_view.c    main_samview       390 bam_destroy1(b);
  h sam_view.c    main_bam2fq        560 bam_destroy1(b);
  i stats.c       main_stats        1504 bam_destroy1(bam_line);
\end{minted}

\subsubsection{Testing}
...particularly difficult.


%%%%%%%%%%%%%%%%%%%%%%%%%%%%%%%%%%%%%%%%%%%%%%%%%%%%%%%%%%%%%%%%%%%%%%%%%%%%%%%
\section{Analysis Pipeline}

\subsection{samtools mpileup}
\subsection{samtools merge}
\subsection{bcftools call}

%%%%%%%%%%%%%%%%%%%%%%%%%%%%%%%%%%%%%%%%%%%%%%%%%%%%%%%%%%%%%%%%%%%%%%%%%%%%%%%
\chapter{Current Status}

